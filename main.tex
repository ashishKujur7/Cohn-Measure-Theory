\documentclass[12pt]{article}
\usepackage[margin=1in]{geometry}
\usepackage{amsfonts, amsmath}
\usepackage[T1]{fontenc}
\usepackage{mathrsfs, enumitem}
\usepackage{dirtytalk,hyperref}
\usepackage[utf8]{inputenc}
\usepackage{amssymb}
\usepackage{amsfonts}
\usepackage{amsmath}
\usepackage{amsthm}
\usepackage{color}
\usepackage{hyperref}
\usepackage{csquotes}
\usepackage{fourier}

\newtheorem{theorem}{Theorem}[subsection]
\newtheorem{lemma}[theorem]{Lemma}
\newtheorem{claim}[theorem]{Claim}
\newtheorem{proposition}[theorem]{Proposition}
\newtheorem{corollary}[theorem]{Corollary}
\newtheorem{fact}[theorem]{Fact}
\newtheorem{notation}[theorem]{Notation}
\newtheorem{observation}[theorem]{Observation}
\newtheorem{conjecture}[theorem]{Conjecture}

\theoremstyle{definition}
\newtheorem{definition}[theorem]{Definition}
\newtheorem{example}[theorem]{Example}

\theoremstyle{remark}
\newtheorem{remark}[theorem]{Remark}
\theoremstyle{plain}
\newcommand{\ignore}[1]{}

% section symbol
\renewcommand{\thesection}{\S\arabic{section}}

% \renewcommand{\Pr}{{\bf Pr}}
% \newcommand{\Prx}{\mathop{\bf Pr\/}}
% \newcommand{\E}{{\bf E}}
% \newcommand{\Ex}{\mathop{\bf E\/}}
% \newcommand{\Var}{{\bf Var}}
% \newcommand{\Varx}{\mathop{\bf Var\/}}
% \newcommand{\Cov}{{\bf Cov}}
% \newcommand{\Covx}{\mathop{\bf Cov\/}}

% shortcuts for symbol names that are too long to type
\newcommand{\eps}{\epsilon}
\newcommand{\lam}{\lambda}
\renewcommand{\l}{\ell}
\newcommand{\la}{\langle}
\newcommand{\ra}{\rangle}
\newcommand{\wh}{\widehat}
\newcommand{\wt}{\widetilde}

% % "blackboard-fonted" letters for the reals, naturals etc.
\newcommand{\R}{\mathbb R}
\newcommand{\N}{\mathbb N}
\newcommand{\Z}{\mathbb Z}
\newcommand{\F}{\mathbb F}
\newcommand{\Q}{\mathbb Q}
\newcommand{\C}{\mathbb C}

% % operators that should be typeset in Roman font
% \newcommand{\poly}{\mathrm{poly}}
% \newcommand{\polylog}{\mathrm{polylog}}
% \newcommand{\sgn}{\mathrm{sgn}}
% \newcommand{\avg}{\mathop{\mathrm{avg}}}
% \newcommand{\val}{{\mathrm{val}}}

% % complexity classes
% \renewcommand{\P}{\mathrm{P}}
% \newcommand{\NP}{\mathrm{NP}}
% \newcommand{\BPP}{\mathrm{BPP}}
% \newcommand{\DTIME}{\mathrm{DTIME}}
% \newcommand{\ZPTIME}{\mathrm{ZPTIME}}
% \newcommand{\BPTIME}{\mathrm{BPTIME}}
% \newcommand{\NTIME}{\mathrm{NTIME}}

% values associated to optimization algorithm instances
\newcommand{\Opt}{{\mathsf{Opt}}}
\newcommand{\Alg}{{\mathsf{Alg}}}
\newcommand{\Lp}{{\mathsf{Lp}}}
\newcommand{\Sdp}{{\mathsf{Sdp}}}
\newcommand{\Exp}{{\mathsf{Exp}}}

% if you think the sum and product signs are too big in your math mode; x convention
% as in the probability operators
\newcommand{\littlesum}{{\textstyle \sum}}
\newcommand{\littlesumx}{\mathop{{\textstyle \sum}}}
\newcommand{\littleprod}{{\textstyle \prod}}
\newcommand{\littleprodx}{\mathop{{\textstyle \prod}}}

% horizontal line across the page
\newcommand{\horz}{
\vspace{-.4in}
\begin{center}
\begin{tabular}{p{\textwidth}}\\
\hline
\end{tabular}
\end{center}
}

% calligraphic letters
\newcommand{\calA}{{\cal A}}
\newcommand{\calB}{{\cal B}}
\newcommand{\calC}{{\cal C}}
\newcommand{\calD}{{\cal D}}
\newcommand{\calE}{{\cal E}}
\newcommand{\calF}{{\cal F}}
\newcommand{\calG}{{\cal G}}
\newcommand{\calH}{{\cal H}}
\newcommand{\calI}{{\cal I}}
\newcommand{\calJ}{{\cal J}}
\newcommand{\calK}{{\cal K}}
\newcommand{\calL}{{\cal L}}
\newcommand{\calM}{{\cal M}}
\newcommand{\calN}{{\cal N}}
\newcommand{\calO}{{\cal O}}
\newcommand{\calP}{{\cal P}}
\newcommand{\calQ}{{\cal Q}}
\newcommand{\calR}{{\cal R}}
\newcommand{\calS}{{\cal S}}
\newcommand{\calT}{{\cal T}}
\newcommand{\calU}{{\cal U}}
\newcommand{\calV}{{\cal V}}
\newcommand{\calW}{{\cal W}}
\newcommand{\calX}{{\cal X}}
\newcommand{\calY}{{\cal Y}}
\newcommand{\calZ}{{\cal Z}}

% bold letters (useful for random variables)
%----------------------------------------------
% \renewcommand{\a}{{\boldsymbol a}}
% \renewcommand{\b}{{\boldsymbol b}}
% \renewcommand{\c}{{\boldsymbol c}}
% \renewcommand{\d}{{\boldsymbol d}}
% \newcommand{\e}{{\boldsymbol e}}
% \newcommand{\f}{{\boldsymbol f}}
% \newcommand{\g}{{\boldsymbol g}}
% \newcommand{\h}{{\boldsymbol h}}
% \renewcommand{\i}{{\boldsymbol i}}
% \renewcommand{\j}{{\boldsymbol j}}
% \renewcommand{\k}{{\boldsymbol k}}
% \newcommand{\m}{{\boldsymbol m}}
% \newcommand{\n}{{\boldsymbol n}}
% \renewcommand{\o}{{\boldsymbol o}}
% \newcommand{\p}{{\boldsymbol p}}
% \newcommand{\q}{{\boldsymbol q}}
% \renewcommand{\r}{{\boldsymbol r}}
% \newcommand{\s}{{\boldsymbol s}}
% \renewcommand{\t}{{\boldsymbol t}}
% \renewcommand{\u}{{\boldsymbol u}}
% \renewcommand{\v}{{\boldsymbol v}}
% \newcommand{\w}{{\boldsymbol w}}
% \newcommand{\x}{{\boldsymbol x}}
% \newcommand{\y}{{\boldsymbol y}}
% \newcommand{\z}{{\boldsymbol z}}
% \newcommand{\A}{{\boldsymbol A}}
% \newcommand{\B}{{\boldsymbol B}}
% \newcommand{\C}{{\boldsymbol C}}
% \newcommand{\D}{{\boldsymbol D}}
% \newcommand{\E}{{\boldsymbol E}}
% \newcommand{\F}{{\boldsymbol F}}
% \newcommand{\G}{{\boldsymbol G}}
% \renewcommand{\H}{{\boldsymbol H}}
% \newcommand{\I}{{\boldsymbol I}}
% \newcommand{\J}{{\boldsymbol J}}
% \newcommand{\K}{{\boldsymbol K}}
% \renewcommand{\L}{{\boldsymbol L}}
% \newcommand{\M}{{\boldsymbol M}}
% \renewcommand{\O}{{\boldsymbol O}}
% \renewcommand{\P}{{\mathbb{P}}}
% \newcommand{\Q}{{\boldsymbol Q}}
% \newcommand{\R}{{\boldsymbol R}}
% \renewcommand{\S}{{\boldsymbol S}}
% \newcommand{\T}{{\boldsymbol T}}
% \newcommand{\U}{{\boldsymbol U}}
% \newcommand{\V}{{\boldsymbol V}}
% \newcommand{\W}{{\boldsymbol W}}
% \newcommand{\X}{{\boldsymbol X}}
% \newcommand{\Y}{{\boldsymbol Y}}
% \newcommand{\Z}{{\boldsymbol Z}}

% script letters
\newcommand{\scrA}{{\mathscr A}}
\newcommand{\scrB}{{\mathscr B}}
\newcommand{\scrC}{{\mathscr C}}
\newcommand{\scrD}{{\mathscr D}}
\newcommand{\scrE}{{\mathscr E}}
\newcommand{\scrF}{{\mathscr F}}
\newcommand{\scrG}{{\mathscr G}}
\newcommand{\scrH}{{\mathscr H}}
\newcommand{\scrI}{{\mathscr I}}
\newcommand{\scrJ}{{\mathscr J}}
\newcommand{\scrK}{{\mathscr K}}
\newcommand{\scrL}{{\mathscr L}}
\newcommand{\scrM}{{\mathscr M}}
\newcommand{\scrN}{{\mathscr N}}
\newcommand{\scrO}{{\mathscr O}}
\newcommand{\scrP}{{\mathscr P}}
\newcommand{\scrQ}{{\mathscr Q}}
\newcommand{\scrR}{{\mathscr R}}
\newcommand{\scrS}{{\mathscr S}}
\newcommand{\scrT}{{\mathscr T}}
\newcommand{\scrU}{{\mathscr U}}
\newcommand{\scrV}{{\mathscr V}}
\newcommand{\scrW}{{\mathscr W}}
\newcommand{\scrX}{{\mathscr X}}
\newcommand{\scrY}{{\mathscr Y}}
\newcommand{\scrZ}{{\mathscr Z}}

\title{Lecture Notes in Measure Theory}
\author{Ashish Kujur}
\date{Last Updated: \today}
\begin{document}
\tableofcontents

\newpage

\section{Measures}
\subsection{Algebras and Sigma-Algebras}
\subsection{Measures}
\subsection{Outer Measures}
\subsection{Lebesgue Measure}
\subsection{Completeness and Regularity}
\subsection{Dynkin Classes}

\newpage

\section{Functions and Integrals}
\subsection{Measurable Functions}

\subsubsection{Question 1}
Observe that all this question wants us to prove is that
\begin{equation*}
    \chi_{\limsup A_n}=\limsup \chi _ {A_n}
    \label{<+label+>}
\end{equation*}
This is easy and simply follows from the definition.

\subsubsection{Question 2}
Let $Y$ be a subset of $\R$ which is not Borel measurable. Then $\chi _ Y$ cannot be Borel measurable (See Example 2.1.2 (b) in the book). Observe that $Y$ cannot be countable for otherwise we could write $Y$ as union of its singleton members of $Y$ and hence $Y$ would be Borel set. Thus, $Y$ is uncountable.
Now, consider the set of functions $J:=\left\{ \chi_{\left\{ y \right\}}: y\in Y \right\}$. It is easy to see that $\chi _{Y} =\sup \left\{ \chi_{\left\{ y \right\}}: y\in Y \right\}$. Clearly, $J$ is set of Borel measurable functions whose supremum is not Borel measurable.

\subsubsection{Question 3}
Let $f: \R \to \R$ be a function which is differentiable everywhere. Consider the sequence of function $\left\{ f_n \right\}_{n\in \N}$ which is given by $f_n \left( x \right) = \frac{f\left( x+ \frac{1}{n} \right)-f\left( x \right) }{1/n}$ for every $x\in\R$. Clearly, $f_n \to f$ pointwise on $\R$ and each $f_n$ is measurable. Since limit of Borel measurable functions is measurable, we have that $f$ is Borel measurable. 

\subsubsection{Question 4}
Let $A:=\left\{ x\in X : \lim_{n} f_n (x) \text{ exists and finite} \right\}$. Let $B:=\left\{ x\in X: \limsup_n f_n (x) =\liminf_n f_n (x) \right\}$. Then $B$ is measurable by virtue of Proposition 2.1.5 and 2.1.3. Note that $B$ is the set of all points $x$ in $X$ where $\lim_n f_n (x)$ exists.  Now $|f|$ is a measurable function. So the set $C:=\bigcap_n \left\{ x\in X : |f| (x) \le n \right\}$ is measurable. Notice that set $A= B\cap C$ and this completes the proof. 

\subsubsection{Question 5}
Let $\left( X, \scrA \right)$ be a measurable space and let $f,g: X\to \R$ be simple, measurable functions. We first show the following statement is true: A measurable function $f: X\to \R$ is simple iff there exists $\alpha _1 , \ldots , \alpha _n \in \R$ and $A_1 , \ldots , A_n \in \scrA$ such that $f=\sum_{i=1}^{n} a_i \chi_{A_i}$.

Let $f: X\to \R$ be measurable. We start the proof of the above statement. $\left( \Longrightarrow \right)$ Suppose that $f$ is simple. Then, by definition, $f$ takes only finitely many values, say, $\alpha _1 , \ldots, \alpha _n$. Since $f$ is measurable, the set $A_i := \left\{ x\in X : f\left( x \right) = \alpha _i \right\}$ is measurable. Then it is easy to check that $f=\sum_{i=1}^{n} a_i \chi_{A_i}$. $(\Longleftarrow)$ Suppose that there exists $\alpha _1 , \ldots , \alpha _n \in \R$ and $A_1 , \ldots , A_n \in \scrA$ such that $f=\sum_{i=1}^{n} a_i \chi_{A_i}$. Suppose that $\alpha_1 < \alpha _2 < \ldots < \alpha _n$. It is fairly easy to check by cases that $f$ is measurable (Just see what happens when $\alpha_i<t$ for each $i$).

\begin{enumerate}
    \item Let $f, g: X \to \R$ be simple measurable functions. Then there exists $\alpha _1 , \ldots , \alpha _n \in \R$ and $A_1 , \ldots , A_n \in \scrA$ such that $f=\sum_{i=1}^{n} a_i \chi_{A_i}$ and 
	there exists $\beta _1 , \ldots , \beta _m$ and measurable sets $B_1 , B_2 , \ldots , B_m$ such that $g=\sum_{j=1}^{m} b_j \chi_{B_j}$. It is then easy to see that $f+g = \sum_{i}\sum_{j} \left( a_i + b_j \right) \chi_{A_i \cap B_j}$ and $fg =\sum_{i}\sum_{j} a_i b_j \chi _{A_i \cap B_j}$. By equivalent statement that we proved above, we are done.

    \item Let $f, g: X\to \R$ be measurable functions. Then by Proposition 2.1.8, there exist a nondecreasing sequences of real valued measurable simple functions $\left\{ f_n \right\}$ and $\left\{ g_n \right\}$ converging pointwise to $f$ and $g$ respectively. By part (1) of this problem $\left\{ f_n + g_n \right\}$ is a nondecresasing sequence of real valued measurable simple functions converging pointwise to $f+g$. By Proposition 2.1.5, we have that limits of measurable functions are measurable and we are done.

\end{enumerate}

\subsubsection{Question 6}
Let $f: X \to \R$ be measurable function. We first show that $x\mapsto t -f (x)$ is a measurable function for every $t\in\R$. Let $t\in \R$ be fixed, let $s\in \R$ be some arbitrary real number. Then $\left\{ x\in X : t-f(x) <s \right\}= \left\{ x\in X: f(x)>t-s \right\}$. Since the set on the right hand side of the previous inequality is measurable as $f$ is measurable, we are done.

Now, let $f,g : X \to \mathbb R$ be measurable functions. To show that $f+g$ is measurable, let $t\in\R$. Then the set $\left\{ x\in X : f(x)+g(x) <t \right\}= \left\{ x\in X: g(x) <t-f(x) \right\}$. Since $g$ is measurable and the function $x\mapsto t-f(x)$ is measurable by the previous paragraph, Proposition 2.1.6 implies that $f+g$ is measurable.

\subsubsection{Question 7}

\subsubsection{Question 8}

\subsubsection{Question 9}

\subsubsection{Question 10}
Let $\scrV _0$ be the collection of all Borel functions $f:\R \to \R$. Proposition 2.1.7. shows that $\scrV _ 0$ is a vector space. Since every continuous function is Borel measurable, we have the set of allcontinuous functions from $\R$ to $\R$ is contained in $\scrV _0$. The third property is satisfied due to Proposition 2.1.5.

We follow the hint given in the book. Let $\scrV$ be a collection of functions satisfying the three criteria given. Let $S\left( \scrV \right)=\left\{ A\subseteq \R : \chi_{A} \in \scrV \right\}$. Clearly


\newpage

\subsection{Properties That Hold Almost Everywhere}
\subsubsection{Question 1}
Consider the function $\chi _{\R\setminus\Q}$ and the constant function $1$. Both are Borel measurable. and they both agree on a dense set of $\R$, namely, $\R\setminus \Q$. But the set $\left\{ x\in\R : \chi_{\R \setminus \Q} \ne 1 \right\}=\Q$ which is a countable set and hence $\lambda$-negligible.

\subsubsection{Question 2}
Let $\left\{ x_n \right\}$ be a sequence of real numbers and let $\mu$ be the measure defined by $\mu = \sum_n \delta_{x_n}$. Exercise 1.2.6 shows that $\mu$ is indeed a measure on $\left( \R , \scrB \left( \R \right) \right)$. Let $f,g : \R \to \R $ be two functions.

We want to show that $f,g$ agree $\mu$-almost everywhere iff $f(x_n)=g(x_n)$ holds for every $n$.

$\left( \Longrightarrow \right)$ Suppose that $f,g$ agree almost everywhere. Then $A:=\left\{ x\in \R : f(x)\ne g(x) \right\}$ is $\mu$ negligible. Since $A \in \scrB (\R)$, we have that $\mu (A) =0$. So if there was some $k\in \N$ such that $f(x_k)\not =g(x_k)$ then $x_k\in A$ and then $\mu \left( \left\{ x_k \right\} \right) =1$. But since $\left\{ x_k \right\} \subseteq A$, we have that $1 < 0$ which is absurd. Hence, we are done!

$\left( \Longleftarrow \right)$ Suppose that $f(x_n ) = g(x_n)$ for every $n\in\N$. We need to prove that $A$ is $\mu$-negligible. Since $A \cap \left\{ x_k : k\in \N \right\} = \emptyset$, we have that $\mu (A) = \sum _n \delta _{x_n} \left( A \right) =0$.

\subsubsection{Question 3}
We show the following is true: If $f$ is a continuous real-valued function on $\R$ and $f=0$ $\lambda$-almost everywhere then $f$ is the zero function.

By assumption, $\mu\left( \left\{ x\in \R : f\left( x \right)\ne 0 \right\}\right) =0$.

Let $x\in \R$. We claim that for every $\varepsilon > 0$ there is some $y\in \R$ such that $f(y)=0$ and $|y-x|<\varepsilon$.

Suppose not. Then there is some $\varepsilon >0$ such that for every $y\in \R$ with $|y-x|<\varepsilon$ we have $f(y)\ne 0$.

But note that previous statement is a contradiction as $\left( x-\varepsilon, x+\varepsilon \right) \subset \left\{ x\in \R : f(x) \ne 0 \right\}$.
\end{document}
